\documentclass[a4paper,11pt]{article}
  % Métadonnées
  \title{Génération procédurale de contenu personnalisé pour créer une expérience de jeu adaptée au style de jeu et aux traits de personnalité du joueur}
  \author{Lecoq Simon (LECS09129600)}
  \date{}
  % Packages
  \usepackage{lmodern,amsmath}
  \usepackage[hyphens]{url}
  \usepackage{tabularx,array,arydshln}
  \usepackage{tikz}
  \usetikzlibrary{shapes.misc,arrows.meta,positioning,calc}
  % Corps du document
  \begin{document}
    % Titre
    \maketitle
    % Style
    \setlength{\parskip}{1em}

    % I - CONTEXTE
    \section{Contexte}
    
      % Courte introduction sur les jeux vidéo
      L'industrie des jeux vidéo est un secteur d'activité en plein essor et qui continue de s'agrandir. 
      Estimé à une valeur de 78 milliards de dollars en 2017, le marché des jeux vidéo devrait atteindre les 90 milliards d'ici 2020 avec 2.5 milliards de joueurs (en incluant les joueurs occasionnels)\footnote{2018 Video Game Industry Statistics, Trends \& Data : \url{wepc.com/news/video-game-statistics/}}.
      Démocratisés dans les années 1980 grâce avec l'apparition des bornes d'arcades et l'arrivée des premières consoles de jeux, et plus encore avec l'avènement des jeux mobiles au début des années 2010 grâce aux smartphones, les jeux vidéo font désormais partie intégrante de la culture populaire.
    
      % Couts de développement et ressources
      Toutefois, s'il était possible auparavant d'émerveiller les joueurs avec une poignée de personnes talentueuses, l'attente du public n'a cessé de croître avec les avancées technologiques. 
      Il n'est pas rare que le développement d'un jeu vidéo nécessite le travail de plusieurs dizaines, voire centaines de personnes, afin mener à bien la conception, le développement, la direction artistique, le doublage, le markéting, etc.
      Inéluctablement, les coûts de production ont eux aussi augmentés, parfois même allant jusqu'à rivaliser avec l'industrie hollywoodienne : 
      la réalisation de \textit{Grand Theft Auto V} (Rockstar Games, 2013), qui est le fruit de 5 années de travail mobilisant plus de 300 personnes, possède un coût total estimé à 250 millions de dollars (dont 137 millions pour le développement)\footnote{Développement de \textit{GTA V} : \url{fr.wikipedia.org/wiki/D\%C3\%A9veloppement_de_Grand_Theft_Auto_V}}.

      % Public cible élargi
      Pour compenser les coûts importants et rentabiliser leurs produits, la solution la plus évidente pour les producteurs de jeux vidéo est d'élargir le public cible.
      Afin d'étendre la démographie potentiellement intéressée, il convient de prendre en considération l'âge, le genre, la personnalité, la culture, etc. des joueurs.
      Par exemple, dans \textit{Splatoon 2} (Nintendo, 2017), les terrains et modes de jeux suivent une rotation imposée par les développeurs ce qui satisfait les joueurs japonais mais qui peut frustrer les joueurs occidentaux en y voyant ici une contrainte\footnote{Un concepteur de Nintendo explique pourquoi le \textit{Salmon Run} n'est pas toujours disponible : \url{kotaku.com/nintendo-designer-explains-why-salmon-run-isnt-always-a-1801065018}}. 
      L'exemple présenté illustre le fait qu'il est difficile de pouvoir satisfaire entièrement un large public aussi varié sans adapter certains aspects du jeu pour correspondre aux attentes des joueurs.
          
      % Pose le problème sur lequel va porter la question de recherche
      La situation actuelle ne permet donc plus la création de nouveau contenu original à la cadence espérée par les joueurs et pour tous les types de joueurs, en raison des goulots d'étranglements liés aux manques de ressources humaines, financières ou temporelles \cite{Hendrikx}.
      Ainsi, les concepteurs de jeux vidéo tout comme les joueurs se retrouvent confrontés à ce problème.
      
      % Introduction de la génération procédurale de contenu
      La PCG (<< Procedural Content Generation >>, ou génération procédurale de contenu) peut potentiellement être utilisée afin de combler le déficit de ressources du côté des créateurs de jeux.
      La PCG permet d'automatiser la création de contenu par un ensemble de règles régies par des algorithmes. 
      Si l'on se concentre sur les applications de la PCG en jeux vidéo, elle peut notamment être utilisé pour générer de nouvelles mécaniques de jeu, des quêtes, des armes, des dialogues, des environnements, des textures, etc., tout ceci dépendant ultimement de comment les concepteurs choisissent de l'implémenter et de s'en servir.
      Par exemple, dans \textit{No Man's Sky} (Hello Games, 2016), la PCG fait partie intégrante des mécaniques de jeux et est utilisé afin de générer un univers immense contenant plusieurs trillions de planètes à explorer, chacune peuplée d'une faune et d'une flore générés également de façon procédurale\footnote{Site officiel de \textit{No Man's Sky} : \url{nomanssky.com}}. 
      Le contenu disponible aurait débouché sur des coûts prohibitifs s'il avait été entièrement réalisé à la main, cependant la PCG a permis de limiter de façon significative le nombre de ressources nécessaire à la création de celui-ci.

      % Introduction de la personnalisation dans les jeux
      Reste que si le contenu générable est supposément infini grâce à la PCG, il n'est utile que s'il plaît aux joueurs, l'un des buts premiers des jeux vidéos étant le divertissement.
      Cette lourde responsabilité est attribué aux concepteurs eux-mêmes, toutefois l'utilisation de la PCG requiert de faire << confiance >> à l'ordinateur en lui déléguant une partie du processus de création \cite{Riedl}.  
      Pour s'assurer que le contenu généré plaît au joueur, une des solutions envisageables est de s'inspirer du concept de << jeu personnalisé >>.
      En somme, cela consiste à réaliser un jeu qui exploite des modèles du joueur afin d'optimiser l'expérience de jeu du joueur en question \cite{Bakkes}.
      La personnalisation individuelle devrait (à condition d'être correctement exécutée) résulter sur des << meilleurs >> jeux avec des joueurs plus engagés, mieux divertis et globalement plus satisfaits \cite{Bakkes}.
      
    % II - QUESTION DE RECHERCHE
    \section{Question de recherche}

      % Objet de la recherche
      L'objectif de cet article est de proposer une solution permettant aux concepteurs de jeux vidéo d'offrir plus de contenu à partir de moins de ressources tout en satisfaisant une majorité de joueurs.
      Ce travail s'inscrit dans la lignée des travaux portant sur l'ED-PCG (<< Experience-Driven PCG>>) \cite{Yannakakis}, qui consiste à utiliser la génération procédurale de contenu en synergie avec l'adaptation de contenu pour optimiser l'expérience de jeu de l'utilisateur.
      
      % Partie 1 : PCG
      Avant de continuer, il convient de présenter succinctement les SB-PCG (<< Search-Based PCG >>), qui sont un sous-ensemble des algorithmes de PCG. 
      Ceux-ci utilisent une fonction permettant d'attribuer une évaluation au contenu généré afin de guider les recherches au sein du domaine de contenu générable.
      La flexibilité fournie par l'utilisation d'une FE (Fonction d'Évaluation) font des SB-PCG une méthode de choix pour la génération de contenu de l'ED-PCG. 
      Le tableau \ref{table:fitness-functions} en page \pageref{table:fitness-functions} répertorie les différentes FE couramment utilisées en SB-PCG. 
      
      % Partie 2 : PEM
      En ce qui concerne la modélisation de l'expérience de jeu du joueur, notée PEM (<< Player Experience Modeling >>), il existe trois approches qui sont les suivantes : subjective, objective et interactive \cite{Yannakakis}.
      Les deux premières se focalisent sur l'aspect affectif et cognitif du joueur, tandis que la troisième mets l'accent sur l'aspect comportemental et cognitif du joueur.
      Ces méthodes ne sont pas exclusives l'une de l'autre et peuvent être combinées.      
      Le tableau \ref{table:pems} en page \pageref{table:pems} en fait l'inventaire. 
      
      % Limitation du champ de recherche (1)
      Il est possible de constater que le domaine de l'ED-PCG est plutôt vaste, et il convient donc de limiter le champ des recherches.
      Dans l'optique de remplir le mieux possible l'objectif présenté au début de cette section, les contraintes supplémentaires suivantes sont examinées :

      \begin{enumerate}
        \vspace{-1em}
        \item La modélisation du joueur doit se perfectionner au cours du temps.
        \vspace{-0.8em}
        \item Les données doivent être collectées de façon non intrusive.
        \vspace{-0.8em}
        \item Les données collectées
        \vspace{-1em}
      \end{enumerate}

      % Limitation du champ de recherche (2)
      Ces restrictions permettent d'écarter les SB-PCG avec une FE basée sur la simulation statique (1) et celles avec des FE interactives, jugées trop intrusive ou imprécises (2, 3).
      La PEM subjective n'est pas traitée pour les mêmes raisons (2, 3).
      De nombreuses recherches sur les FE directes ont été réalisées, et dans le but d'étudier des solutions non explorées, celles-ci ne sont pas prises en compte pour les sections méthodologie et expérimentation.

      % Aboutissement de la question de recherche
      Ainsi, les recherches présentées ici se concentrent principalement sur les ED-PCG basée sur la SB-PCG avec FE simulée dynamique avec une PEM objective ou interactive. 

      % Taxinomie des SB-PCG
      \begin{table}
        \begin{tabularx}{\linewidth}{p{2.2cm} X}
          \hline
            \multicolumn{2}{c}{\textbf{Taxinomie des fonctions d'évaluations pour la SB-PCG}} \\
          \hline
            \textbf{Directe} & Basé sur les caractéristiques du contenu généré \\
            ~~\textit{Théorie} & \textit{- évaluées selon les normes des concepteurs.} \\ 
            ~~\textit{Données} & \textit{- évaluées selon des données collectées.} \\
          \hline
            \textbf{Simulée} & Basé sur les performances d'un agent intelligent par rapport au contenu généré \\
            ~~\textit{Statique} & \textit{- en supposant que l'agent ne change pas au cours du jeu.} \\
            ~~\textit{Dynamique} & \textit{- en supposant que l'agent évolue au cours du jeu.} \\
          \hline
            \textbf{Interactive} & Basé sur la collecte de données lié au contenu généré \\
            ~~\textit{Explicite} & \textit{- récupérées en questionnant directement le joueur.} \\
            ~~\textit{Implicite} & \textit{- récupérées en arrière-plan de manière transparente.} \\
          \hline
        \end{tabularx}
        \caption{Les trois catégories de fonctions d'évaluation présente dans la taxinomie de Togelius \cite{Togelius}. Chaque catégorie est elle-même divisée en deux sous-catégories, indiquées en écriture italique.}
        \label{table:fitness-functions}
      \end{table}
        
      % Taxinomie des PEM
      \begin{table}
        \begin{tabularx}{\linewidth}{p{2.2cm} X}
          \hline
            \multicolumn{2}{c}{\textbf{Approches de modélisation de l'expérience de jeu d'un joueur}} \\
          \hline
            \textbf{Subjective} & Données collectées par les retours du joueur sur son expérience de jeu (e.g. questionnaires, discussions, etc.). \\
          \hline
            \textbf{Objective} & Données collectées sur les réactions du joueur par des sources tierces (e.g. caméra, cardiofréquencemètre, etc.). \\
          \hline
            \textbf{Interactive} \footnotesize<<Gameplay-based>> & Données collectées via les interactions entre le jeu et le joueur (e.g. fréquence d'utilisation d'une certaine fonctionnalité, etc.). \\
          \hline
        \end{tabularx}
        \caption{Les trois approches permettant de modéliser l'expérience de jeu d'un joueur repertoriées par Yannakakis \cite{Yannakakis}.}
        \label{table:pems}
      \end{table}

    % III - MOTIVATIONS
    \section{Motivations}

    % IV - CHALLENGES
    \section{Challenges}
      
    % V - CONTRIBUTIONS
    \section{Contributions}

    % RÉFÉRENCES
    \begin{thebibliography}{9}
      \bibitem{Hendrikx}
        Hendrikx, M., Meijer, S., Van Der Velden, J., \& Iosup, A. (2013). Procedural content generation for games. ACM Transactions on Multimedia Computing, Communications, and Applications, 9(1), 1-22. doi:10.1145/2422956.2422957.
      \bibitem{Riedl}
        Riedl, M. O. (2010). Scalable personalization of interactive experiences through creative automation. Comput. Entertain., 8(4), 1-3. doi:10.1145/1921141.1921146
      \bibitem{Bakkes} 
        Bakkes, S., Tan, C. T., \& Pisan, Y. (2012). Personalised Gaming: A motivation and Overview of Literature. Paper presented at the Proceedings of The 8th Australasian Conference on Interactive Entertainment: Playing the System, Auckland, New Zealand. 
      \bibitem{Yannakakis}
        Yannakakis, G. N., \& Togelius, J. (2011). Experience-Driven Procedural Content Generation. IEEE Transactions on Affective Computing, 2(3), 147-161. doi:10.1109/t-affc.2011.6
      \bibitem{Togelius}
        Togelius, J., Yannakakis, G. N., Stanley, K. O., \& Browne, C. (2011). Search-Based Procedural Content Generation: A Taxonomy and Survey. IEEE Transactions on Computational Intelligence and AI in Games, 3(3), 172-186. doi:10.1109/tciaig.2011.2148116
    \end{thebibliography}

  \end{document}



